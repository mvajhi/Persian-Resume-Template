%-------------------------------------------------------------------------------
%	SECTION TITLE
%-------------------------------------------------------------------------------
\cvsection{منتخب پروژه‌ها}


%-------------------------------------------------------------------------------
%	CONTENT
%-------------------------------------------------------------------------------
\begin{cventries}

%---------------------------------------------------------
  \cventry
    {\lr{AI \& Machine Learning Course Projects}} % Title
    {\lr{Link: \href{https://github.com/mvajhi/Introduction-to-Artificial-Intelligence-and-Machine-Learning}{GitHub}}} % Institution
    {\lr{University of Tehran}} % Location
    {۱۴۰۳} % Date(s)
    {
      \begin{cvitems}
        \item {\lr{Genetic Algorithms, Hidden Markov Models, Bayesian Networks}}
        \item {\lr{Unsupervised \& Supervised Learning (K-Means, DBSCAN, SVM, Random Forest, XGBoost, Regression, Decision Trees)}}
        \item {\lr{Deep Learning (CNN, 1D CNN for text), Transfer Learning (VGG16), NLP (Word2Vec)}}
        \item {\lr{Data preprocessing (EDA, feature engineering, PCA, tokenization, stemming)}}
        \item {\lr{Model evaluation (Accuracy, F1, AUC, GridSearchCV)}}
        \item {\lr{Reinforcement Learning (Q-learning, MDP)}}
        \item {\lr{Tools: Python, PyTorch, Keras, Scikit-learn, NumPy, Pandas, Jupyter Notebook, Google Colab}}
        \item {\lr{Real-world datasets (Boston Housing, image/text data)}}
      \end{cvitems}
    }
%---------------------------------------------------------
  \cventry
    {\lr{Data Science Course Projects}}
    {\lr{Link: \href{https://github.com/mvajhi/Introduction-to-Data-Science}{GitHub}}}
    {\lr{University of Tehran}}
    {۱۴۰۳}
    {
      \begin{cvitems}
        \item {\lr{Web scraping, data cleaning, EDA, visualization (BeautifulSoup, Selenium, Pandas, Matplotlib)}}
        \item {\lr{Monte Carlo simulation, Central Limit Theorem, statistical hypothesis testing (NumPy, SciPy)}}
        \item {\lr{Linear regression, OLS, gradient descent, K-Fold cross-validation (scikit-learn, manual implementation)}}
        \item {\lr{Feature engineering, handling missing data, unit conversion, multi-currency analysis}}
        \item {\lr{Unsupervised learning: PCA, K-Means, DBSCAN, clustering evaluation}}
        \item {\lr{Semi-supervised learning: Label Propagation, LLM-based auto-labeling (Phi-3)}}
        \item {\lr{Sentiment analysis, real-world data modeling, neural networks, SVR, decision trees (TensorFlow/Keras, scikit-learn)}}
        \item {\lr{Team collaboration, practical insight extraction, large-scale data handling (Spark)}}
      \end{cvitems}
    }
%---------------------------------------------------------
  \cventry
    {\lr{DevOps Course Projects}}
    {\lr{Link: \href{https://github.com/mvajhi/DevOps_Lab}{GitHub}}}
    {\lr{Arvan Academy}}
    {۱۴۰۴}
    {
      \begin{cvitems}
        \item {\lr{Vagrant and VirtualBox provisioning with shell scripts}}
        \item {\lr{Automated Nginx installation and configuration using Ansible}}
        \item {\lr{Docker multi-stage builds for minimal Go application images}}
        \item {\lr{IP Table and firewall configuration via shell scripting}}
        \item {\lr{Basic Nginx server setup on virtual machines}}
        \item {\lr{Traefik reverse proxy setup and Docker Compose configuration}}
        \item {\lr{Automated Docker installation with Ansible}}
        \item {\lr{Repository server deployment and orchestration (Traefik, MinIO, Nexus) automated by Ansible \& Docker}}
      \end{cvitems}
    }
%---------------------------------------------------------
  \cventry
    {\lr{Internet Engineering Course Projects}}
    {\lr{Link: \href{https://github.com/mvajhi/internet-engineering}{GitHub}}}
    {\lr{University of Tehran}}
    {۱۴۰۴}
    {
      \begin{cvitems}
        \item {\lr{Hotel Management System: Java-based application for managing hotel bookings, customers, and rooms. Focused on validation and extensive unit testing. (Java, Jackson, JUnit)}}
        \item {\lr{Book Shop Management System (CLI): Console app for bookshop management—users, books, authors, borrowing, and sales. (Java, Jackson, JUnit)}}
        \item {\lr{Book Shop Management System (REST API): RESTful API for managing books, users, and transactions. (Java, Spring Boot, JUnit, Postman)}}
        \item {\lr{Book Shop Management System (Frontend): Added a React frontend to the bookshop REST API. (Java, Spring Boot, React, JUnit, Postman)}}
        \item {\lr{Book Shop Management System (DB): Integrated MySQL database for persistent storage in the bookshop system. (Java, Spring Boot, React, JUnit, Postman, MySQL)}}
      \end{cvitems}
    }
%---------------------------------------------------------
\cventry
    {\lr{Advanced Programming Projects (C++)}}
    {\lr{Link: \href{https://github.com/mvajhi/advance-programming-UT}{GitHub}}}
    {\lr{University of Tehran}} % Organization - Assuming these were university projects
    {Fall 2022} % Date(s) - Adjust as per your actual timeline for these projects
    {
      \begin{cvitems}
        \item {\lr{\textbf{Course Schedule Management System:} Developed a system to automate weekly course scheduling for high school, managing teacher availability and course requirements. Focused on top-down design and modularity.}}
        \item {\lr{\textbf{Employee Payroll and HR Management System:} Designed an object-oriented system to manage employee/team data, working hours, salaries, and bonuses from CSV inputs. Implemented comprehensive reporting and data modification functionalities with error handling.}}
        \item {\lr{\textbf{Turtix Game Clone:} Developed a 2D platformer game using SFML, focusing on event-driven and object-oriented programming. Implemented core game mechanics, collision detection, item collection, and state transitions across multiple scenes.}}
        \item {\lr{\textbf{Driver Mission Management System:} Designed a polymorphic system for managing driver missions (time-based, distance-based, count-based) in a ride-sharing application. Supported mission assignment, ride completion tracking, and detailed driver reports, emphasizing inheritance and polymorphism.}}
        \item {\lr{\textbf{Fantasy Football League System:} Developed a backend system for a fantasy football league, handling real-world data parsing, complex scoring (sigmoid normalized), user authentication, player transfers, and captain selection. Emphasized OOD and robust error handling.}}
        \item {\lr{\textbf{Online Quiz System:} Developed a CLI-based quiz system supporting various question types (single/multiple-choice, short-answer). Implemented answer validation, immediate feedback, and comprehensive quiz reports with final grade calculation, designed with polymorphism.}}
      \end{cvitems}
    }

%---------------------------------------------------------
\cventry
  {\lr{Compiler Implementation for a Custom Programming Language}} % Project Title
  {\lr{Link: \href{https://github.com/mvajhi/Complier-Design-and-Programming-Languages-Design}{GitHub}}}
  {\lr{University of Tehran}} % Organization (Assuming it's a university project)
  {Fall 2024} % Date(s) - Adjust as per your actual timeline for this project
    {
      \begin{cvitems}
        \item {\lr{\textbf{Description:} Designed and implemented a complete compiler for a custom programming language, covering Lexical, Syntax, Semantic Analysis, and Code Generation.}}
        \item {\lr{\textbf{Technologies:} Java (Primary), ANTLR 4 (Lexer/Parser), AST, Symbol Table, Visitor Pattern.}}
        \item {\lr{\textbf{Key Responsibilities:} Designed language grammar with ANTLR 4, constructed AST, and implemented lexical/syntax error detection. Implemented Symbol Table for scope/type management; performed name analysis and semantic error checking (e.g., type consistency, re-declarations). (If applicable) Translated AST into intermediate/target code (e.g., Jasmin/JVM bytecode) and implemented optimization passes.}}
      \end{cvitems}
    }

%---------------------------------------------------------
\cventry
  {\lr{Statistical and Probabilistic Data Analysis in Engineering Projects}} % Project Title
  {\lr{Link: \href{https://github.com/mvajhi/Probability_and_Statistics_in_Engineering}{GitHub}}}
  {\lr{University of Tehran}} % Organization (Assuming it's a university project)
  {Fall 2023} % Date(s) - Adjust as per your actual timeline for this project
    {
      \begin{cvitems}
        \item {\lr{\textbf{Book Ratings Analysis:} Performed EDA, data cleaning, and applied regression/classification models on book rating datasets; investigated statistical distributions.}}
        \item {\lr{\textbf{General Statistical Problems:} Implemented advanced probability/statistics concepts via coding, including statistical simulations and hypothesis testing (e.g., t-test, chi-squared, ANOVA).}}
        \item {\lr{\textbf{Digital Image & General Data Analysis:} Analyzed handwritten digit images (e.g., PCA for classification) and performed statistical analysis on the Tarbiat dataset for pattern identification.}}
        \item {\lr{\textbf{FIFA Player Data Analysis:} Cleaned and analyzed FIFA 2020 player data, assessing attributes, identifying correlations, and comparing distributions via visualization.}}
        \item {\lr{Authored detailed analytical reports (\texttt{report.pdf}, \texttt{description.pdf}) for each project phase, documenting methodologies and findings.}}
        \item {\lr{\textbf{Technologies Used:} Python, NumPy, Pandas, Matplotlib, Seaborn, Scikit-learn, Jupyter Notebook.}}
      \end{cvitems}
    }

%---------------------------------------------------------
% \cventry
%   {\lr{Hardware for AI Projects}} % Project Title
%   {\lr{Link: \href{https://github.com/mvajhi/Hardware-for-AI}{GitHub}}} % GitHub link
%   {\lr{University of Tehran}} % Organization
%     {Fall 2024} % Date(s) - Adjust as per your actual timeline for these projects
%     {
%       \begin{cvitems}
%         \item {\lr{\textbf{CNN Implementation for CIFAR-10 Classification:} Designed and implemented a Convolutional Neural Network (CNN) in TensorFlow/Keras for image classification on the CIFAR-10 dataset. Utilized data preprocessing and augmentation techniques to enhance model robustness and generalization.}}
%         \item {\lr{\textbf{MobileNet Quantization Analysis:} Conducted comprehensive analysis of MobileNet quantization techniques, including Post-Training Quantization (PTQ), to optimize model size and inference speed for resource-constrained AI hardware. Leveraged TensorFlow Lite to implement and evaluate various strategies, quantifying trade-offs between accuracy and compression ratios.}}
%         \item {\lr{\textbf{LSQ+ for Learned Step Size Quantization:} Implemented and applied Learned Step Size Quantization Plus (LSQ+), an advanced hardware-aware training technique, to optimize deep learning models for extreme low-bit precision. Developed custom quantization layers and achieved significant model compression and efficiency gains while maintaining high accuracy at very low bit-widths.}}
%       \end{cvitems}
%     }

%---------------------------------------------------------
\cventry
  {\lr{Analysis of Mongolian Vocabulary in Persian Literature}} % Project Title
  {\lr{Link: \href{https://github.com/mvajhi/mogoli-word-after-and-before-war}{GitHub}}} % GitHub link
  {\lr{University of Tehran}} % Organization
    {Fall 2022} % Date(s) - Adjust as per your actual timeline for these projects
    {
    \begin{cvitems}
        \item {\lr{\textbf{Description:} Analyzed changes in the usage of Mongolian vocabulary in Persian literature before and after the Mongol invasion, focusing on the impact of Mongol presence on the Persian language through comparative analysis of literary works from pre- and post-Mongol periods.}}
      \end{cvitems}
    }

%---------------------------------------------------------

\cventry
  {\lr{CCNP-level Network Lab Projects}} % Project Title
  {\lr{Link: \href{https://github.com/mvajhi/CCNP_lab}{GitHub}}} % GitHub link
  {\lr{University of Tehran}} % Organization (Assuming these were university lab projects)
    {Fall 2023 - Spring 2024} % Date(s) - Adjust as per your actual timeline for these labs
    {
      \begin{cvitems}
        \item {\lr{\textbf{Advanced Routing Protocol Implementation:} Designed, implemented, and troubleshooted complex routing protocols (EIGRP in Named Mode, OSPFv2 including Stub/NSSA Areas and Virtual Links) in intricate network topologies. Configured Route Summarization and Redistribution with Metric Type control.}}
        \item {\lr{\textbf{Layer 2 Switching & High Availability:} Managed Layer 2 switching and enhanced network redundancy using EtherChannel (LACP Passive/Active). Implemented High Availability protocols like HSRP and GLBP (with Preemption and Priority settings) for gateway failover and load balancing, integrated with IP SLA for link tracking.}}
        \item {\lr{\textbf{IP Services & Secure Connectivity:} Configured DHCP Server (Layer 3 Switch) and DHCP Relay. Implemented Static NAT and PAT for network address translation. Established secure and private communications using GRE Tunnel and IPsec VPN (with ISAKMP Policy and IPsec Transform-Set). Utilized Policy-Based Routing (PBR) with IP SLA tracking for advanced traffic steering and automated failover.}}
        \item {\lr{\textbf{Infrastructure Security:} Enhanced network infrastructure security by configuring Zone-Based Firewall (ZBFW) with Class-Map and Policy-Map for network segmentation and policy enforcement. Secured device access using SSH.}}
        \item {\lr{\textbf{Technologies Used:} Cisco IOS, EIGRP, OSPFv2, HSRP, GLBP, EtherChannel, NAT, PAT, IPsec VPN, ZBFW, Spanning Tree (MST), VLANs, Inter-VLAN Routing, IP SLA. (Simulator: Cisco Packet Tracer / GNS3)}}
      \end{cvitems}
    }
%---------------------------------------------------------
\cventry
  {\lr{CS50xAI Projects}} % Project Title
  {\lr{Link: \href{https://github.com/mvajhi/AI.cs50xiran}{GitHub}}} % GitHub link
  {\lr{Harvard University (via CS50x)}} % Organization
    {Dec 2022} % Date(s) - Adjust if needed, based on your CS50x completion date
    {
      \begin{cvitems}
        \item {\lr{\textbf{Six Degrees of Kevin Bacon AI:} Implemented a Breadth-First Search (BFS) graph traversal algorithm in Python to find the shortest connection paths between actors in the IMDb database.}}
        \item {\lr{\textbf{Knights and Knaves Logic Solver:} Developed a system using propositional logic and a knowledge base to solve complex logical puzzles and determine character identities through logical inference.}}
        \item {\lr{\textbf{Minesweeper AI:} Designed and implemented an AI for the Minesweeper game that uses logical inference and pattern recognition to identify safe moves and locate mines.}}
        \item {\lr{\textbf{Tic-Tac-Toe AI:} Developed an AI for the Tic-Tac-Toe game using the Minimax algorithm to determine the optimal move for winning or achieving a draw.}}
        \item {\lr{\textbf{Technologies Used:} Python, Graph Algorithms, BFS, Propositional Logic, Knowledge Representation, Logical Inference, Minimax Algorithm, Game AI, Game Theory.}}
      \end{cvitems}
    }

\end{cventries}
